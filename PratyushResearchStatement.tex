% !TeX program = lualatex
%%%%%%%%%%%%%%%%%%%%%%%%%%%%%%%%%%%%%%%%%
% "ModernCV" CV and Cover Letter
% LaTeX Template
% Version 1.1 (9/12/12)
%
% This template has been downloaded from:
% http://www.LaTeXTemplates.com
%
% Original author:
% Xavier Danaux (xdanaux@gmail.com)
%
% License:
% CC BY-NC-SA 3.0 (http://creativecommons.org/licenses/by-nc-sa/3.0/)
%
% Important note:
% This template requires the moderncv.cls and .sty files to be in the same
% directory as this .tex file. These files provide the resume style and themes
% used for structuring the document.
%
%%%%%%%%%%%%%%%%%%%%%%%%%%%%%%%%%%%%%%%%%

%----------------------------------------------------------------------------------------
%	PACKAGES AND OTHER DOCUMENT CONFIGURATIONS
%----------------------------------------------------------------------------------------

\documentclass[11pt, a4paper]{article} % Font sizes: 10, 11, or 12; paper 
%sizes: a4paper, letterpaper, a5paper, legalpaper, executivepaper or landscape; 
%font families: sans or roman

%\moderncvstyle{classic} % CV theme - options include: 'casual' (default), 
%%'classic', 'oldstyle' and 'banking'

\usepackage{float}
\usepackage{graphics}
\usepackage{graphicx}
\usepackage{palatino}
\usepackage{caption}
\usepackage{xcolor}
\usepackage[hidelinks]{hyperref}

% character encoding
%\usepackage[utf8]{inputenc}                  % if you are not using xelatex ou 
%lualatex, replace by the encoding you are using
%\usepackage{CJKutf8}                         % if you need to use CJK to 
%%typeset your resume in Chinese, Japanese or Korean
\definecolor{burgundy}{HTML}{800020}
%\moderncvcolor{burgundy}   
%\usepackage{fontspec}
%\renewcommand*\familydefault{\sfdefault}
\usepackage[T1]{fontenc}
%\usepackage{tgheros}
%\usepackage[style=verbose]{biblatex}
\usepackage{natbib}

%\renewcommand{\refname}{References}

%\moderncvcolor{burgundy} % CV color - options include: 'blue' (default), 
%%'orange', 'green', 'red', 'purple', 'grey' and 'black'
\usepackage[scale=0.78, textheight=720pt]{geometry} % Reduce document margins
%\setlength{\hintscolumnwidth}{3cm} % Uncomment to change the width of the 
%%dates column
%\setlength{\makecvtitlenamewidth}{10cm} % For the 'classic' style, uncomment 
%%to adjust the width of the space allocated to your name

%----------------------------------------------------------------------------------------
%	NAME AND CONTACT INFORMATION SECTION
%----------------------------------------------------------------------------------------

%\firstname{\LARGE{Pratyush}} % Your first name
%\familyname{\LARGE{Kumar}} % Your last name

%\author{Pratyush Kumar}

% All information in this block is optional, comment out any lines you don't 
%need
%\title{Pratyush Kumar \\ Research Statement: Machine learning tools 
%for process operations and control}
%\address{785 Camino Del Sur, Apt 220}{}
%\mobile{(302) 584 3464}
%\phone{(000) 111 1112}
%\fax{(000) 111 1113}
%\email{ethan.eagle@gmail.com}
%\homepage{staff.org.edu/~jsmith}{staff.org.edu/$\sim$jsmith} % The first 
%%argument is the url for the clickable link, the second argument is the url 
%%displayed in the template - this allows special characters to be displayed 
%%such 
%%as the tilde in this example
%\extrainfo{additional information}
%\photo[70pt][0.4pt]{pictures/picture} % The first bracket is the picture 
%%height, the second is the thickness of the frame around the picture (0pt for 
%%no 
%%frame)
%\quote{"A witty and playful quotation" - John Smith}

%----------------------------------------------------------------------------------------

\begin{document}

\noindent \large{Pratyush Kumar}
\vspace{0.1in} \\
Research summary: Machine learning tools for process operations and control
\normalsize	

% Print the CV title
%----------------------------------------------------------------------------------------
%	EDUCATION SECTION
%----------------------------------------------------------------------------------------
\vspace{0.1in}
%\hline
\section*{Introduction}

I am passionate about developing innovative technologies for the energy 
efficient and economically optimal operation of industrial systems. For my 
graduate research, I am developing new algorithms to leverage machine learning 
methods for feedback control of dynamical systems. An advancement in 
the state-of-the-art automation algorithms may lead to the deployment of 
advanced control methods on a broad class of industrial systems, 
therefore, provide monetary benefits for industries in commercial applications 
and enable energy efficient operation of industrial processes.

Over the past few decades, algorithmic advancements in system identification, 
optimal control, and optimization have led to the implementation of 
advanced computer-aided control strategies in a wide range of industrial 
sectors. Model predictive control (MPC) is currently the leading feedback 
control technology implemented in several industries for the operation of 
multi-variable and constrained processes. The MPC controller solves an 
optimization problem in real-time that is formulated based on a dynamic model 
and operational objective of the process. Recent progresses in machine 
learning \citep{lecun:bengio:hinton:2015}, improved data acquisition 
capabilities in 
several industries, 
and 
increased available computing power, present an opportunity to advance the 
process modeling and optimization methods used for deployment of the MPC 
technology.

The goal of my doctoral research is to build process modeling and online 
optimization methods using new machine learning tools. I classify 
my projects in two categories: a) the use of neural networks (NNs) for 
fast online execution of MPC than currently possible with optimization solvers, 
and b) data-driven and first principles based mathematical modeling of 
dynamical systems. The success of projects in these two areas enables 
the deployment of the MPC technology in a broad range of industrial sectors and 
improves the performance of the MPC controller in current or future industrial 
implementations.

\section*{Fast online execution of model predictive control using neural 
networks}

The optimal solution of the MPC optimization problem can be characterized as a 
fixed function over the state-space of parameters in the optimization problem. 
This function is known as the MPC feedback law in the literature 
\citep{bemporad:morari:dua:pistikopoulos:2002} that is typically a complex 
nonlinear function, challenging to explicitly 
store and evaluate online for the MPC controller execution. Neural networks are 
a class of function approximators well known within the machine learning field 
for their ability to represent complex nonlinear functions. I am studying the 
capability of NNs to approximate the MPC feedback law such that NNs can be used 
online to compute control actions in place of using optimization solvers. The 
MPC feedback law can be approximated offline using a NN in the following two 
steps: (i) generate optimal control solutions using an optimization solver 
for some set of parameters in the MPC optimization problem corresponding to 
anticipated plant operational scenarios, and (ii) solve a NN training 
optimization problem using the generated set of parameters and optimal control 
solutions. The NN obtained after this training step can be used as the feedback 
controller online during the plant operation in place of using an optimization 
solver.

The deployment of MPC using linear models requires solutions of quadratic 
programs (QP) in real-time. I have demonstrated in 
\cite*{kumar:rawlings:wright:2021} that NNs can be used to replace online QPs 
in large-scale linear MPC applications. In the paper, we proposed a structured 
NN architecture that is suitable to achieve offset-free performance in the 
closed-loop, and illustrated that NNs can be used to speed-up real-time 
computations by four orders of magnitude compared to an available QP solver. 
This result encourages an investigation to examine the capabilities of NNs to 
approximate the MPC feedback law for nonlinear, economic, and mixed-integer MPC 
problems. I am currently performing simulation studies to study the potential 
of NNs to approximate the MPC feedback law for a nonlinear MPC formulation with 
complex chemical engineering examples.

\section*{Data-driven and first principles based system identification}

The construction of dynamic process models is a vital step in the operation and 
control of industrial processes. I am developing systematic process modeling 
frameworks to build data-driven and first principles based hybrid
models for use in process optimization and control. An actuator-to-sensor 
model can be constructed in the following steps: (i) perform system 
identification tests by exciting the process with a varying manipulated input 
signal and collect the observed actuator and sensor measurements for training, 
(ii) identify available first principles knowledge and postulate a hybrid model 
using NNs to represent some unknown functions such as reaction rate laws, and 
(iii) estimate parameters in the postulated model by solving an optimization 
problem that minimizes the prediction error of the model with the collected 
data.

A caveat in using NNs as a part of the actuator-to-sensor dynamic model is that 
the subsequent use of the overall model in optimization results in a non-convex 
optimization problem to be solved in real time. We are currently investigating 
methods to address this limitation. We propose the following hierarchical 
control strategy: (i) use the identified dynamic model to determine optimum 
operating conditions by solving a steady state optimization problem, (ii) 
linearize the model around the obtained optimum steady state to compute a 
linear model for subsequent use in an MPC controller. Since a linear model is 
used in  MPC, solutions of QPs is required in real time by the MPC controller. 
The steady state optimization problem remains non-convex, however, we emphasize 
that this optimization problem can be solved infrequently relative to the MPC 
execution frequency. We have compelling simulation results with this 
hierarchical operational strategy on chemical process examples. We are 
currently preparing an article for publication on this project.

In some industrial applications, enough first principles knowledge is available 
about the plant such that a NN is not required in the actuator-to-sensor model. 
However, the plant could be affected by some noticeable measured or unmeasured 
disturbances with specific patterns in time. Commercial buildings are an 
example of such application, where accurate thermal dynamic models can be 
constructed but the buildings are affected by unmeasured disturbances such as 
heat generated by occupants in the building, electrical equipment, etc. In such 
cases, NNs can be used to construct forecasts of the disturbances for use in 
MPC optimization problems. Since NNs are only used to make disturbance 
forecasts, the MPC optimization problem is not a challenging non-convex 
optimization problem. I am performing simulation studies to demonstrate the 
application of NNs for disturbance modeling in this class of industrial systems.

\section*{Future research interests}

I am interested to pursue a career as a research scientist at a US national 
laboratory and develop new machine learning and statistics based 
feedback control methods. I believe my PhD research has given me the 
experience, background, and skills to investigate and develop new control 
technologies. I would be excited to pursue the research projects already 
defined for the post-doctoral position or I am also interested to collaborate 
with researchers at the laboratory and pursue the following research ideas.

%\subsection*{Fast mixed-integer MPC with neural networks}
%
%In some applications, the manipulated actuators can only take some 
%discrete values rather than a continuum of values in a constraint set. For such 
%applications, it is advantageous to consider the discrete actuators as integer 
%decision variables in the MPC optimization problem, and solve mixed-integer 
%programs online for MPC deployment. The start-of-the-art mixed-integer 
%programming solvers can get slow, however, for moderate to large sized MPC 
%problems. I propose to study the applications of NNs to approximate solutions 
%of mixed-integer programs, by designing the NNs offline based on the solutions 
%of mixed-integer problems, so that NNs can be used in real-time as a 
%replacement for  mixed-integer optimization problems.
%
%\subsection*{Image-based model predictive control}
%
%With the growing data acquisition capabilities in industries, real-time image 
%measurements are becoming prevalent in several industries such as autonomous 
%vehicles in the self-driving car industry or crystallization processes in the 
%chemical industry. The current MPC technology utilizes measurements from 
%sensors that provide numerical values of the measured quantity. Image 
%measurements may contain more information about the process being controlled 
%compared with a typical numeric sensor. I am interested to develop algorithms 
%that combine image measurements with typical numeric sensors to provide 
%improved feedback information for model and optimization based control methods.
%
%%\subsection*{Model-based reinforcement learning with hybrid process models}
%%
%%Black-box machine learning models of process systems are not interpretable and 
%%they often provide no intuition about the internal model structure used to 
%%make 
%%the final prediction. The models often require large training data sets, and 
%%can become impractical to train in industrial applications. A hybrid process 
%%modeling framework is preferred over the Black-box approach due 
%%interpretability and maintainability. Hybrid models can be developed by 
%%combining first-principles based knowledge about the plant with neural 
%%networks. The parameters (weights) in the neural network component of the 
%%model 
%%can be estimated from data. The hybrid models can still be sometimes difficult 
%%to use in process optimization, due to non-convexity in the optimization with 
%%the use of neural networks. However, the developed hybrid model can be used to 
%%generate data for a Q-learning algorithm and subsequently obtain an 
%%approximate 
%%controller for online implementation.
%
%\subsection*{Stability theory of approximate neural network controllers}
%
%While several researchers have proposed the use of neural networks to 
%approximate the MPC feedback law for fast online implementation of MPC, the 
%stability theory of neural networks remains an open issue. I am interested in 
%the development of algorithms to obtain stability by design in neural network 
%controllers. 
%
%\subsection*{Nonlinear state estimation using neural networks}
%
%State estimation of nonlinear systems is an important research area with broad 
%industrial impact in performance monitoring and output-feedback MPC 
%implementation. Moving horizon estimation (MHE) is an optimization based method 
%for state estimation in nonlinear systems. A nonlinear optimization problem is 
%solved at every sample time to estimate the state of the system from 
%measurements. Approximating this nonlinear optimization problem with neural 
%networks has been proposed in . I am interested in investigating the benefits 
%and limitations of this approach as well as use recently developed theory in 
%the MHE literature to establish theory for approximate neural network based MHE 
%estimators.
%
%\subsection*{Noise covariance estimation methods for improved state estimation}
%
%State estimation of nonlinear systems is an important research area with broad 
%industrial impact in performance monitoring and output-feedback MPC 
%implementation. Moving horizon estimation (MHE) is an optimization based method 
%for state estimation in nonlinear systems. A nonlinear optimization problem is 
%solved at every sample time to estimate the state of the system from 
%measurements. Approximating this nonlinear optimization problem with neural 
%networks has been proposed in . I am interested in investigating the benefits 
%and limitations of this approach as well as use recently developed theory in 
%the MHE literature to establish theory for approximate neural network based MHE 
%estimators.

\bibliographystyle{abbrvnat}
\bibliography{articles, books, proceedings, resgrppub}

\end{document}